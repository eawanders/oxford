% Options for packages loaded elsewhere
\PassOptionsToPackage{unicode}{hyperref}
\PassOptionsToPackage{hyphens}{url}
\documentclass[
]{article}
\usepackage{xcolor}
\usepackage[margin=1in]{geometry}
\usepackage{amsmath,amssymb}
\setcounter{secnumdepth}{-\maxdimen} % remove section numbering
\usepackage{iftex}
\ifPDFTeX
  \usepackage[T1]{fontenc}
  \usepackage[utf8]{inputenc}
  \usepackage{textcomp} % provide euro and other symbols
\else % if luatex or xetex
  \usepackage{unicode-math} % this also loads fontspec
  \defaultfontfeatures{Scale=MatchLowercase}
  \defaultfontfeatures[\rmfamily]{Ligatures=TeX,Scale=1}
\fi
\usepackage{lmodern}
\ifPDFTeX\else
  % xetex/luatex font selection
\fi
% Use upquote if available, for straight quotes in verbatim environments
\IfFileExists{upquote.sty}{\usepackage{upquote}}{}
\IfFileExists{microtype.sty}{% use microtype if available
  \usepackage[]{microtype}
  \UseMicrotypeSet[protrusion]{basicmath} % disable protrusion for tt fonts
}{}
\makeatletter
\@ifundefined{KOMAClassName}{% if non-KOMA class
  \IfFileExists{parskip.sty}{%
    \usepackage{parskip}
  }{% else
    \setlength{\parindent}{0pt}
    \setlength{\parskip}{6pt plus 2pt minus 1pt}}
}{% if KOMA class
  \KOMAoptions{parskip=half}}
\makeatother
\usepackage{longtable,booktabs,array}
\usepackage{calc} % for calculating minipage widths
% Correct order of tables after \paragraph or \subparagraph
\usepackage{etoolbox}
\makeatletter
\patchcmd\longtable{\par}{\if@noskipsec\mbox{}\fi\par}{}{}
\makeatother
% Allow footnotes in longtable head/foot
\IfFileExists{footnotehyper.sty}{\usepackage{footnotehyper}}{\usepackage{footnote}}
\makesavenoteenv{longtable}
\usepackage{graphicx}
\makeatletter
\newsavebox\pandoc@box
\newcommand*\pandocbounded[1]{% scales image to fit in text height/width
  \sbox\pandoc@box{#1}%
  \Gscale@div\@tempa{\textheight}{\dimexpr\ht\pandoc@box+\dp\pandoc@box\relax}%
  \Gscale@div\@tempb{\linewidth}{\wd\pandoc@box}%
  \ifdim\@tempb\p@<\@tempa\p@\let\@tempa\@tempb\fi% select the smaller of both
  \ifdim\@tempa\p@<\p@\scalebox{\@tempa}{\usebox\pandoc@box}%
  \else\usebox{\pandoc@box}%
  \fi%
}
% Set default figure placement to htbp
\def\fps@figure{htbp}
\makeatother
\setlength{\emergencystretch}{3em} % prevent overfull lines
\providecommand{\tightlist}{%
  \setlength{\itemsep}{0pt}\setlength{\parskip}{0pt}}
\usepackage{bookmark}
\IfFileExists{xurl.sty}{\usepackage{xurl}}{} % add URL line breaks if available
\urlstyle{same}
\hypersetup{
  pdftitle={Week 5: Differences in Differences},
  pdfauthor={Your name},
  hidelinks,
  pdfcreator={LaTeX via pandoc}}

\title{Week 5: Differences in Differences}
\author{Your name}
\date{HT 2025}

\begin{document}
\maketitle

\section{Introduction}\label{introduction}

The intuition of the DiD strategy is to combine two simpler approaches.
The first difference (before and after) \emph{eliminates unit-specific
fixed effects}. Then, the second difference \emph{eliminates time fixed
effects}. With this approach, we can, under some assumptions, obtain an
unbiased estimate of a (policy) intervention.

We learned that we can break down the difference between treated and
untreated units in post-treatment as the Average Treatment Effect
Amongst the Treated (ATT), different time trends and selection bias.
However, we can impose additional estimation assumptions to retrieve a
credible estimate of this treatment effect. The key assumption in
diff-in-diff studies is the so-called \textbf{Parallel Trends} or
\textbf{Common Trends} assumption. This assumption states that in the
absence of the treatment/policy, we should expect to see the treated and
untreated units following similar trends over time. Unfortunately, we
cannot comprehensively test whether this assumption holds, but we can at
least conduct some analyses that provide indicative, although not
sufficient, evidence for the parallel trends assumption to hold.

We also learned that we can calculate the ATT in different ways.
Specifically, we learned that we can manually calculate the
difference-in-difference estimator.

\begin{itemize}
\tightlist
\item
  Group-period interactions
\item
  Unit and time dummies and a treatment indicator
\end{itemize}

We rely on the parallel trends assumption by assuming that time-trends
are the same for both treated and untreated units.

Finally, we discussed inference and, in particular, standard errors.
Specifically, we use panel data, meaning repeated observations over
time, to estimate diff-in-differences models. Panel data likely exhibits
serially correlated regressors and residuals. As a result, it's
important to adjusted our coefficients' standard errors, e.g.~by using
\emph{clustered standard errors}.

\section{Lab Session Overview}\label{lab-session-overview}

In this seminar, we will cover the following topics:

\begin{enumerate}
\def\labelenumi{\arabic{enumi}.}
\item
  ``Manually'' calculate the difference-in-differences estimator
\item
  Obtain the difference-in-differences estimator using the \texttt{lm()}
  function
\item
  Check for parallel trends before the treatment occurred
\item
  Use fixed effects in difference-in-differences estimations using both
  the \texttt{lm\_robust()} and \texttt{plm()} function.
\item
  Conduct a placebo test
\end{enumerate}

\textbf{Before proeceeding further:}

\begin{enumerate}
\def\labelenumi{\arabic{enumi}.}
\item
  Create a folder called \texttt{Lab5}.
\item
  Download the data and the empty RMarkdown file from Canvas
\item
  Save both in our \texttt{Lab5} folder.
\item
  Open the RMarkdown file.
\item
  Set your working directory using the \texttt{setwd()} function or by
  clicking on "More".
\end{enumerate}

\section{Refugee exposure and support for the extreme
right}\label{refugee-exposure-and-support-for-the-extreme-right}

Dinas et
al.~\href{https://www.cambridge.org/core/journals/political-analysis/article/waking-up-the-golden-dawn-does-exposure-to-the-refugee-crisis-increase-support-for-extremeright-parties/C50A127CC517968F2D0FA42A2A23FF85}{(2019)})
studied the following question: \emph{Did the influx of refugees in
Greece increase support for the right-wing Golden Dawn party in 2015?}

The authors exploit that the Aegean islands close to the Turkish border
experienced sudden and drastic increases in the number of Syrian
refugees while other islands slightly farther away---but with otherwise
similar institutional and socio-economic characteristics---did not.

The below figure shows exposure to refugees across the Aegean islands:

\pandocbounded{\includegraphics[keepaspectratio]{dinas_et_al_map.png}}

\subsection{Estimating DiD}\label{estimating-did}

\textbf{Group-period interactions}: Here the treatment variable is equal
to 1 for all the years since the unit received the treatment. Then, our
coefficient of interest is captured by the interaction between the
treatment and the time variable. For the example of the Golden Dawn, we
can write the respective diff-in-diffs regression equation (including
the interaction term's coefficient that captures the causal effect) as
follows:

\[gdper_{mt} = \beta_0 + \beta_1eventr_m + \beta_2post_t + \beta_3evertr_m \times post_t + u_{mt}\]

\begin{longtable}[]{@{}
  >{\raggedright\arraybackslash}p{(\linewidth - 4\tabcolsep) * \real{0.1613}}
  >{\centering\arraybackslash}p{(\linewidth - 4\tabcolsep) * \real{0.2903}}
  >{\centering\arraybackslash}p{(\linewidth - 4\tabcolsep) * \real{0.5484}}@{}}
\toprule\noalign{}
\begin{minipage}[b]{\linewidth}\raggedright
\end{minipage} & \begin{minipage}[b]{\linewidth}\centering
\textbf{Post = 0}
\end{minipage} & \begin{minipage}[b]{\linewidth}\centering
\textbf{Post = 1}
\end{minipage} \\
\midrule\noalign{}
\endhead
\bottomrule\noalign{}
\endlastfoot
\textbf{Treat = 0} & \(\beta_0 + u_{mt}\) &
\(\beta_0 + \beta_2 + u_{mt}\) \\
\textbf{Treat = 1} & \(\beta_0 +\beta_1 + u_mt\) &
\(\beta_0 + \beta_1  + \beta_2 + \beta_3 + u_{mt}\) \\
\end{longtable}

Specifically, we can obtain our estimate by calculating the difference
of the outcome variable for both treated and untreated units and then
subtract them from each other:
\[((\beta_0 + \beta_1  + \beta_2 + \beta_3 + u_{mt}) - (\beta_0 + \beta_1 + u_{mt})) -((\beta_0 + \beta_2 + u_{mt})-(\beta_0 + u_{mt}))\]
\[= (\beta_2 + \beta_3) -(\beta_2)\] \[=\beta_3\]

\textbf{Unit and time dummies and a treatment indicator}: Alternatively,
we can represent this estimation strategy for the Golden Dawn example by
the following regression model:

\[gdper_{mt} =  \beta_1treatment_{mt} + \alpha_m \text{unit dummy} + \gamma_t \text{time dummy} +  u_{mt}\]
For treated before treated \(Treatment = 0\):
\[gdper_{mt} =  \beta_1\times 0 + \alpha_m \times 1 + \gamma_t \times +  u_{mt}\]
\[gdper_{mt} = \alpha_m + \gamma_t + u_{mt}\] For treated after treated
\(Treatment = 1\):
\[gdper_{mt} = \beta_1 +  \alpha_m + \gamma_t + u_{mt}\] Then, we can
take the difference before and after:
\[(\beta_1 +  \alpha_m + \gamma_t + u_{mt}) - (\alpha_m + \gamma_t + u_{mt})\]
\[ = \beta_1\]

\section{Dinas et al.~(2019)}\label{dinas-et-al.-2019}

\subsection{Context}\label{context}

\begin{itemize}
\tightlist
\item
  The refugee crisis started in the spring of 2015.
\item
  Greece held an \emph{election in September 2015}, right after the
  \emph{first wave of refugee} arrivals. The variable \texttt{post} is
  coded as 1 for observations from this election and the variable
  \texttt{year}, somewhat unintuitively, as \texttt{2016}.
\item
  The \emph{previous election} had taken place only eight months prior
  in \emph{January 2015}, which was before a significant number of
  refugees arrived. The \texttt{year} variable is coded as \texttt{2015}
  for observations from this election.
\item
  Two \emph{additional elections} were held in 2012 and 2013. The
  \texttt{year} variable captures observations for the election in May
  2012 with the value \texttt{2012} and the election in June 2013 with
  \texttt{2013}.
\item
  The units of analysis are \emph{municipalities}.
\end{itemize}

Let's first familiarise ourselves with the data. A description of some
of the variables that we will use today is below:

\begin{longtable}[]{@{}
  >{\raggedright\arraybackslash}p{(\linewidth - 2\tabcolsep) * \real{0.2609}}
  >{\raggedright\arraybackslash}p{(\linewidth - 2\tabcolsep) * \real{0.7391}}@{}}
\toprule\noalign{}
\begin{minipage}[b]{\linewidth}\raggedright
Variable
\end{minipage} & \begin{minipage}[b]{\linewidth}\raggedright
Description
\end{minipage} \\
\midrule\noalign{}
\endhead
\bottomrule\noalign{}
\endlastfoot
\texttt{Year} & Election year (2012 to 2016) \\
\texttt{municipality} & Municipality id \\
\texttt{post} & A dummy variable that is 1 for the 2016 election and 0
otherwise \\
\texttt{treatment} & A dummy variable that is 1 if the municipality
received a strong increase in refugees before the 2016 election and 0
otherwise \\
\texttt{evertr} & Treatment variable indicating if the municipality
received refugees ( = 1) or not (= 0) \\
\texttt{gdper} & The Golden Dawn's vote share at the municipality
level \\
\end{longtable}

Now let's load the data from our computer by using the
\texttt{read\_dta()} function. We will call this data frame
\textbf{greekislands}.

Let's start by checking our data as always.

\paragraph{Task 1}\label{task-1}

Use the \texttt{head()} function to familiarise yourself with the data
set.**

\begin{verbatim}
## # A tibble: 6 x 9
##    year  muni treatment distance gdper  post  gddif evertr post2
##   <dbl> <dbl>     <dbl>    <dbl> <dbl> <dbl>  <dbl>  <dbl> <dbl>
## 1  2012     1         0     240.  7.98     0 NA          0     0
## 2  2013     1         0     240.  7.28     0 -0.705      0     0
## 3  2015     1         0     240.  6.36     0 NA          0     1
## 4  2016     1         0     240.  7.62     1  1.25       0     1
## 5  2012     2         0     318.  2.58     0 NA          0     0
## 6  2013     2         0     318.  4.28     0  1.70       0     0
\end{verbatim}

This is what we would expect. Recall that the treat variable indicates
municipalities that are treated at some point - independent of the
timing - while the treatment variable marks municipalities after they
were treated.

\begin{verbatim}
## 
## 2012 2013 2015 2016 
##   96   96   96   96
\end{verbatim}

\begin{verbatim}
## [1] 2012 2013 2015 2016
\end{verbatim}

\begin{verbatim}
## [1] 4
\end{verbatim}

Recall that the \texttt{treat} variable indicates municipalities that
are treated at some point - independent of the timing - while the
\texttt{treatment} variable marks municipalities after they were
treated.

As we can see, the data set covers multiple election years. Before
working with the data, let's make sure we know how many and which
elections are included in the data.

\paragraph{Task 2}\label{task-2}

How many elections, i.e.~\texttt{years} are covered by the data?

\begin{verbatim}
## [1] 2012 2013 2015 2016
\end{verbatim}

Let's have a look at general voting patterns of the \emph{Golden Dawn}
over time, irrespective of treatment status of municipalities.

\paragraph{Task 3}\label{task-3}

Plot the vote share of the Golden Dawn Party (\texttt{gdper}) over time.

There are several ways to do this. For instance, you could plot the
dispersion across municipalities by using the \texttt{boxplot} command
or, alternatively, calculate average values per year. Feel free to pick
the option you deem most appropriate.

\subsection{Differences-in-Differences}\label{differences-in-differences}

Being aware of the general trends of the Golden Dawn's vote share is an
important information about context and the party's history. However, it
cannot tell us anything about the treatment effect we seek to analyse:
The arrival of refugees in some Greek municipalities in the summer of
2015.

A naive observer might propose identifying this effect by looking at the
differences between treated and untreated units in the post-treatment
periods. Would this, however, be an appropriate representation of a
possible treatment effect? It clearly would not! Comparing
post-treatment differences only doesn't allows us to account for
unit/municipality-specific effects and voting patterns. Treatment, after
all, was not assigned randomly. We would not be able to say what the
effect of the treatment is unless we can make a statement about how the
treatment changed the outcome or resulted in the diversion from a
previous trajectory. Using a \emph{differences-in-differences} design
allows us to do so.

\paragraph{Task 4}\label{task-4}

Estimate the treatment effect by calculating differences-in-differences
between 2015 and 2016 using the \texttt{mean()} function.

\begin{quote}
Hint: Calculate the differences between treated and untreated units for
the years 2015 and 2016 first.
\end{quote}

\paragraph{Task 5}\label{task-5}

Estimate the difference-in-difference between 2015 and 2016 using an OLS
regression.

You can run a simple OLS with the interaction term of \emph{treat} and a
dummy variable for the post-treatment period as independent variables.
However, you should restrict the data to the years 2015 and 2016,
specifying
\texttt{data\ =\ greekislands{[}greekislands\$year\textgreater{}=2015,{]}}
as argument for your OLS.

\paragraph{Task 6}\label{task-6}

Estimate the difference-in-differences between 2015 and 2016 using a
regression with unit fixed effects.

\begin{quote}
Hint: You can use \texttt{lm\_robust()}, \texttt{feols()},
\texttt{plm()} or \texttt{lm()} with dummy variables.
\end{quote}

\subsection{Generalised Diff-in-Diff}\label{generalised-diff-in-diff}

Let's now extend our analysis by including all pre-treatment periods in
our analysis. The easiest way to do so is running a two-way fixed
effects regression.

\paragraph{Task 7}\label{task-7}

Estimate the difference-in-differences using a two-way fixed effects
regression with all time periods and \texttt{treatment} as independent
variable.

\paragraph{Task 8}\label{task-8}

Calculate robust standard errors for (i) the plm FE model, (ii) the
two-way FE model and present the regression output in a single table.
Also include the simple OLS model in the table.

\begin{quote}
Note: There is no need to adjust standard errors after using
\texttt{lm\_robust()} as the command automatically does that.
\end{quote}

\subsection{Pararell trends}\label{pararell-trends}

As previously mentioned, we cannot, generally, test whether the parallel
trends assumption holds because we cannot, in principle, observe the
vote share of the Golden Dawn among treated and untreated units for the
\textbf{hypothetical scenario} of the 2015 refugee crisis \textbf{not
occurring}. However, we can examine, either visually or through
statistical analyses, whether the Golden Dawn's vote share trends more
or less uniformly for both treated and untreated units \textbf{before
the treatment}, namely the 2015 refugee crisis, occurred. Indeed, if it
was really the treatment (in our case, refugees arriving on the coasts
of some Greek islands in 2015) that causally affected the Golden Dawn's
vote share and not anything else, then we should observe the Golden
Dawn's vote share moving \textbf{in parallel} for units (in our case,
Greek municipalities) that were either treated or not with refugees in
2015. However, note that detecting such parallel trends \textbf{before
the treatment occurred} is only a \textbf{necessary} but not a
\textbf{sufficient condition} for testing the parallel trends assumption
which (hopefully) illustrates why this assumption is, principally, not
testable. Nevertheless, the necessary condition is better than nothing.
Indeed, at the very least, by examining parallel trends before the
treatment occurred, we allow for the parallel trends assumption to be
potentially \textbf{falsified}.

One way to examine these parallel trends before the treatment occurred
is to plot the outcome variable for both treated and untreated units
overtime, especially before the treatment/intervention occurred. To do
so, we, first, need to calculate the Golden Dawn's mean vote share for
both treated and untreated units. Let's do this with tidyverse code this
time, specifically by using the \texttt{group\_by()} and
\texttt{summarise()} functions.

\paragraph{Task 9}\label{task-9}

Calculate the mean vote share for the Golden Dawn for the treated and
the untreated units for all the elections. Store this into a new data
frame called \texttt{plot.parallel}. Use tidy to create this new data
frame.

The syntax and the description of the functions is provided below:

\begin{longtable}[]{@{}
  >{\raggedright\arraybackslash}p{(\linewidth - 2\tabcolsep) * \real{0.3814}}
  >{\raggedright\arraybackslash}p{(\linewidth - 2\tabcolsep) * \real{0.6186}}@{}}
\toprule\noalign{}
\begin{minipage}[b]{\linewidth}\raggedright
Function
\end{minipage} & \begin{minipage}[b]{\linewidth}\raggedright
Description
\end{minipage} \\
\midrule\noalign{}
\endhead
\bottomrule\noalign{}
\endlastfoot
\texttt{new.data\ \textless{}-\ dataframe} & Assignment operator where
the new data frame will be stored \\
\texttt{group\_by()} & Group observations by a variable or set of
variables \\
\texttt{summarise()} & Creates a new data frame with one ore more rows
of each combination of grouping variables \\
\texttt{mutate()} & Allows to create new variable or modify existing
ones \\
\end{longtable}

Let's now plot these results.

\paragraph{Task 10}\label{task-10}

Plot the parallel trends. Set vote share for the Golden Dawn in the y
axis, year in the x-axis. Connect the data points of the two groups
using a line. Place the legend of your plot at the bottom. Change the
default colour. Below you will find all the functions necessary to
generate this plot. Remember to use the plus sign between functions.

\begin{longtable}[]{@{}
  >{\raggedright\arraybackslash}p{(\linewidth - 2\tabcolsep) * \real{0.5795}}
  >{\raggedright\arraybackslash}p{(\linewidth - 2\tabcolsep) * \real{0.4205}}@{}}
\toprule\noalign{}
\begin{minipage}[b]{\linewidth}\raggedright
Function
\end{minipage} & \begin{minipage}[b]{\linewidth}\raggedright
Description
\end{minipage} \\
\midrule\noalign{}
\endhead
\bottomrule\noalign{}
\endlastfoot
\texttt{ggplot(x\ =\ ,\ y\ =,\ colour\ =\ )} & Map data components into
the graph \\
\texttt{geom\_point()} & To generate a scatterplot \\
\texttt{scale\_color\_manual(values=c("colorname1",\ "colorIwant2"))} &
Replace the default palette \\
\texttt{theme(legend.position\ =\ "bottom")} & To place the legend at
the bottom \\
\texttt{geom\_line(aes(x=year,y=vote,\ color\ =\ condition,\ group\ =\ condition))}
& To connect the dots with lines by group \\
\end{longtable}

\paragraph{Optional:}\label{optional}

Now let's look at the leads to identify any anticipatory effects. Let's
imagine that the Golden Dawn back in 2012 believed that there was going
to be a major humanitarian in the future. Then, they thought that they
could exploit this situation to increase their electoral gains. In that
case, we wouldn't be able to disentangle whether changes in vote share
are due to the previous campaigning efforts on the part of the Golden
Dawn or due to the influx of alyssum seekers to Greece. We can use leads
to identify if there are any anticipatory effects. If we find systematic
differences between treated and untreated units, this would suggest that
units in one or both groups are responding to the treatment before
receiving it.

\paragraph{EXTRA Exercise 1:}\label{extra-exercise-1}

Create dummy year variables equal to 1 for every year and only for the
treated municipality. Call this variable leads, plus the year. For
example, the lead2012, will take value 1 only for treated municipalities
and only for observations of these municipalities in the year 2012. You
can see an example below. use the \texttt{mutate()} function to create
these new variables. You can also use the \texttt{ifelse()} function to
create these dummy variables. The syntax of the \texttt{ifelse()}
function is the following:
\texttt{new\ variable\ =\ ifelse(condition,\ "value\ if\ condition\ is\ met",\ "value\ if\ the\ condition\ is\ not\ met")}.
Create these dummy variables for the elections in 2012, 2013, and 2015.

\subparagraph{EXTRA Exercise 2:}\label{extra-exercise-2}

Conduct the same two-way fixed-effect model that we used before, but
rather than using the \texttt{treatment} variable, replace this variable
with the new leads variables that you created. Ran separate estimations
for each lead. Store the outputs of these regressions into different
objects. Does the evidence suggest that there are any anticipatory
effects? Are the results of these three models statistically
significant? You can use the summary() or screenreg() functions to take
a look at your results.

Now, let's plot all the two-way fixed effects models where we used the
\texttt{plm()} function into a single figure.

\subparagraph{EXTRA Exercise 3:}\label{extra-exercise-3}

Plot the coefficients from the leads models, plus the two-way fixed
model for the 2016 election that you used in Question 7
(``twoway\_FE''). Use the \texttt{plot\_coef()} function to generate
this plot. Add the argument \texttt{scale} and set it equal to
\texttt{TRUE}. Also, include the argument \texttt{robust} and set it
equal to \texttt{TRUE}. In addition to the \texttt{plot\_coefs()}
include the \texttt{coord\_flip()} function. This function will flip the
Cartesian coordinates of the plot, so we have the models (years) in the
x-axis and the coefficients in the y axis. Remember to add plus sign
operator between two functions. You can also add the
\texttt{xlab("Year")} function to a label in the x-axis.

\subsection{Placebo test}\label{placebo-test}

We can conduct a placebo test to evaluate whether the parallel trend
holds. We are trying to show that there will be no statistically
significant difference in the trend between treated and untreated
municipalities.

The steps to conduct are the following:

\begin{enumerate}
\def\labelenumi{\arabic{enumi}.}
\tightlist
\item
  Use data for periods that came before the treatment was
  implemented/happened.
\item
  Create a dummy variable for each before treatment that is equal to 1
  only for that specific year and only for treated units (as we did
  before).
\item
  Estimate the difference-in-difference using \texttt{plm()} or
  \texttt{lm\_robust()} function.
\item
  If you find statistically significant results, this may suggest a
  violation of parallel trends.
\end{enumerate}

\subparagraph{Task 14}\label{task-14}

Drop all the observations of the year of the intervention (2016). Do
this using the \texttt{filter()} function. Then, create a fake treatment
variable and call it \texttt{post2} and set it equal to 1 for all
observations in year 2015. This variable would indicate as the
hypothetical case that the municipality would received refugees in 2015.

\subparagraph{Task 15}\label{task-15}

Conduct the same two-way fixed effect model using the \texttt{lm()} and
use the \texttt{post2} variable. Store all the models in a list and plug
this list inside of the \texttt{modelsummary(list)} function to report
your results. Did you find statistically significant differences between
treated and untreated units pre-treatment?. You can see an example of
how to store multiple models in a list \texttt{list()}. Also, subset the
data so one model will only conduct the placebo test using the 2012 and
2015 elections, and another model only using the observations from the
2013 and 2015 elections.

\end{document}
